\documentclass[]{book}
\usepackage{lmodern}
\usepackage{amssymb,amsmath}
\usepackage{ifxetex,ifluatex}
\usepackage{fixltx2e} % provides \textsubscript
\ifnum 0\ifxetex 1\fi\ifluatex 1\fi=0 % if pdftex
  \usepackage[T1]{fontenc}
  \usepackage[utf8]{inputenc}
\else % if luatex or xelatex
  \ifxetex
    \usepackage{mathspec}
  \else
    \usepackage{fontspec}
  \fi
  \defaultfontfeatures{Ligatures=TeX,Scale=MatchLowercase}
\fi
% use upquote if available, for straight quotes in verbatim environments
\IfFileExists{upquote.sty}{\usepackage{upquote}}{}
% use microtype if available
\IfFileExists{microtype.sty}{%
\usepackage{microtype}
\UseMicrotypeSet[protrusion]{basicmath} % disable protrusion for tt fonts
}{}
\usepackage[margin=1in]{geometry}
\usepackage{hyperref}
\hypersetup{unicode=true,
            pdftitle={Studiehandleiding Module Lead Generation},
            pdfauthor={Witek ten Hove \& Roel Linssen},
            pdfborder={0 0 0},
            breaklinks=true}
\urlstyle{same}  % don't use monospace font for urls
\usepackage{natbib}
\bibliographystyle{apalike}
\usepackage{longtable,booktabs}
\usepackage{graphicx,grffile}
\makeatletter
\def\maxwidth{\ifdim\Gin@nat@width>\linewidth\linewidth\else\Gin@nat@width\fi}
\def\maxheight{\ifdim\Gin@nat@height>\textheight\textheight\else\Gin@nat@height\fi}
\makeatother
% Scale images if necessary, so that they will not overflow the page
% margins by default, and it is still possible to overwrite the defaults
% using explicit options in \includegraphics[width, height, ...]{}
\setkeys{Gin}{width=\maxwidth,height=\maxheight,keepaspectratio}
\IfFileExists{parskip.sty}{%
\usepackage{parskip}
}{% else
\setlength{\parindent}{0pt}
\setlength{\parskip}{6pt plus 2pt minus 1pt}
}
\setlength{\emergencystretch}{3em}  % prevent overfull lines
\providecommand{\tightlist}{%
  \setlength{\itemsep}{0pt}\setlength{\parskip}{0pt}}
\setcounter{secnumdepth}{5}
% Redefines (sub)paragraphs to behave more like sections
\ifx\paragraph\undefined\else
\let\oldparagraph\paragraph
\renewcommand{\paragraph}[1]{\oldparagraph{#1}\mbox{}}
\fi
\ifx\subparagraph\undefined\else
\let\oldsubparagraph\subparagraph
\renewcommand{\subparagraph}[1]{\oldsubparagraph{#1}\mbox{}}
\fi

%%% Use protect on footnotes to avoid problems with footnotes in titles
\let\rmarkdownfootnote\footnote%
\def\footnote{\protect\rmarkdownfootnote}

%%% Change title format to be more compact
\usepackage{titling}

% Create subtitle command for use in maketitle
\newcommand{\subtitle}[1]{
  \posttitle{
    \begin{center}\large#1\end{center}
    }
}

\setlength{\droptitle}{-2em}
  \title{Studiehandleiding Module Lead Generation}
  \pretitle{\vspace{\droptitle}\centering\huge}
  \posttitle{\par}
\subtitle{Isatis Academy}
  \author{Witek ten Hove \& Roel Linssen}
  \preauthor{\centering\large\emph}
  \postauthor{\par}
  \predate{\centering\large\emph}
  \postdate{\par}
  \date{2017-10-07}

\usepackage{booktabs}
\usepackage{amsthm}
\makeatletter
\def\thm@space@setup{%
  \thm@preskip=8pt plus 2pt minus 4pt
  \thm@postskip=\thm@preskip
}
\makeatother

\begin{document}
\maketitle

{
\setcounter{tocdepth}{1}
\tableofcontents
}
\chapter{Isatis Academy}\label{isatis-academy}

Isatis Academy is het opleidings- en onderzoeksintituut van de Isatis
Groep. We zijn een open community van didactische en technische experts
met een passie voor kennisontwikkeling en -deling. Ons doel is onze
interne, zakelijke en maatschappelijke partners te helpen groeien door
prachtige innovatie, boeiend onderwijs en laagdrempelige toegang tot
kennis te bieden.

\chapter{Inleiding}\label{inleiding}

In dit hoofdstuk vind je de doelen, de structuur en het didactisch opzet
van de module

\chapter{Hello World}\label{hello-world}

In dit hoofdstuk vind je informatie over de voorbereiding op de module:

\begin{itemize}
\tightlist
\item
  voorkennis
\item
  hardware
\item
  software
\item
  lesmaterialen.
\end{itemize}

\chapter{Competenties}\label{competenties}

In dit hoofdstuk vind je alles over de eindcompetenties van deze module.

\chapter{Activiteiten}\label{activiteiten}

In dit hoofdstuk vind je een beschrijving van de verschillende
onderwijsactiviteiten.

\section{Weblectures}\label{weblectures}

\section{Workshops}\label{workshops}

\section{Individuele opdrachten}\label{individuele-opdrachten}

\section{Teamopdrachten}\label{teamopdrachten}

\section{Seminars en
bedrijfsbezoeken}\label{seminars-en-bedrijfsbezoeken}

\section{Maatschappelijke
participatie}\label{maatschappelijke-participatie}

\chapter{Final Words}\label{final-words}

We have finished a nice book.

\bibliography{packages.bib,book.bib}


\end{document}
