\documentclass[]{book}
\usepackage{lmodern}
\usepackage{amssymb,amsmath}
\usepackage{ifxetex,ifluatex}
\usepackage{fixltx2e} % provides \textsubscript
\ifnum 0\ifxetex 1\fi\ifluatex 1\fi=0 % if pdftex
  \usepackage[T1]{fontenc}
  \usepackage[utf8]{inputenc}
\else % if luatex or xelatex
  \ifxetex
    \usepackage{mathspec}
  \else
    \usepackage{fontspec}
  \fi
  \defaultfontfeatures{Ligatures=TeX,Scale=MatchLowercase}
\fi
% use upquote if available, for straight quotes in verbatim environments
\IfFileExists{upquote.sty}{\usepackage{upquote}}{}
% use microtype if available
\IfFileExists{microtype.sty}{%
\usepackage{microtype}
\UseMicrotypeSet[protrusion]{basicmath} % disable protrusion for tt fonts
}{}
\usepackage[margin=1in]{geometry}
\usepackage{hyperref}
\hypersetup{unicode=true,
            pdftitle={Studiehandleiding Module Lead Generation},
            pdfauthor={Witek ten Hove \& Roel Linssen},
            pdfborder={0 0 0},
            breaklinks=true}
\urlstyle{same}  % don't use monospace font for urls
\usepackage{natbib}
\bibliographystyle{apalike}
\usepackage{longtable,booktabs}
\usepackage{graphicx,grffile}
\makeatletter
\def\maxwidth{\ifdim\Gin@nat@width>\linewidth\linewidth\else\Gin@nat@width\fi}
\def\maxheight{\ifdim\Gin@nat@height>\textheight\textheight\else\Gin@nat@height\fi}
\makeatother
% Scale images if necessary, so that they will not overflow the page
% margins by default, and it is still possible to overwrite the defaults
% using explicit options in \includegraphics[width, height, ...]{}
\setkeys{Gin}{width=\maxwidth,height=\maxheight,keepaspectratio}
\IfFileExists{parskip.sty}{%
\usepackage{parskip}
}{% else
\setlength{\parindent}{0pt}
\setlength{\parskip}{6pt plus 2pt minus 1pt}
}
\setlength{\emergencystretch}{3em}  % prevent overfull lines
\providecommand{\tightlist}{%
  \setlength{\itemsep}{0pt}\setlength{\parskip}{0pt}}
\setcounter{secnumdepth}{5}
% Redefines (sub)paragraphs to behave more like sections
\ifx\paragraph\undefined\else
\let\oldparagraph\paragraph
\renewcommand{\paragraph}[1]{\oldparagraph{#1}\mbox{}}
\fi
\ifx\subparagraph\undefined\else
\let\oldsubparagraph\subparagraph
\renewcommand{\subparagraph}[1]{\oldsubparagraph{#1}\mbox{}}
\fi

%%% Use protect on footnotes to avoid problems with footnotes in titles
\let\rmarkdownfootnote\footnote%
\def\footnote{\protect\rmarkdownfootnote}

%%% Change title format to be more compact
\usepackage{titling}

% Create subtitle command for use in maketitle
\newcommand{\subtitle}[1]{
  \posttitle{
    \begin{center}\large#1\end{center}
    }
}

\setlength{\droptitle}{-2em}
  \title{Studiehandleiding Module Lead Generation}
  \pretitle{\vspace{\droptitle}\centering\huge}
  \posttitle{\par}
\subtitle{Isatis Academy}
  \author{Witek ten Hove \& Roel Linssen}
  \preauthor{\centering\large\emph}
  \postauthor{\par}
  \predate{\centering\large\emph}
  \postdate{\par}
  \date{2017-10-08}

\usepackage{booktabs}
\usepackage{amsthm}
\makeatletter
\def\thm@space@setup{%
  \thm@preskip=8pt plus 2pt minus 4pt
  \thm@postskip=\thm@preskip
}
\makeatother

\begin{document}
\maketitle

{
\setcounter{tocdepth}{1}
\tableofcontents
}
\chapter{Welkom}\label{welkom}

\section{Isatis Academy}\label{isatis-academy}

Isatis Academy is het opleidings- en onderzoeksintituut van de Isatis
Groep. We zijn een open community van didactische en technische experts
met een passie voor kennisontwikkeling en -deling. Ons doel is onze
interne, zakelijke en maatschappelijke partners te helpen groeien door
prachtige innovatie, boeiend onderwijs en laagdrempelige toegang tot
kennis te bieden.

\section{Module Lead Generations (LG)}\label{module-lead-generations-lg}

\subsection{Lead Generation}\label{lead-generation}

Het doel van \emph{Lead Generation} is klanten optimaal ondersteunen bij
hun online aankoopbeslissingen.

\subsection{Voor wie?}\label{voor-wie}

Je bent nieuwsgierig naar online gedrag van mensen en hoe ze reageren op
verschillende impulsen. Binnen het LG-team kun je verschillende rollen
vervullen. Echter voor een aantal gebieden ben jij de expert en hiermee
draag je bij aan de unieke identiteit van het team en jullie product.

\subsection{Rollen}\label{rollen}

Een LG-team bestaat uit de volgende rollen met de volgende
verantwoordelijkheden:

\begin{table}

\caption{\label{tab:unnamed-chunk-1}Overzicht van competenties en levels voor Content Marketeer}
\centering
\begin{tabular}[t]{ll}
\toprule
Rollen & Verantwoordelijkheden\\
\midrule
*Online Marketeer* & SEO Data\\
 & Zoekwoorden in tekst\\
*Content Marketeer* & Communicatie\\
 & Interviews\\
 & Social Media uitingen\\
\addlinespace
 & Doelgroep communicatie\\
*Developer (front -end)* & HTML\\
 & CSS\\
 & Vormgeving\\
*Designer * & Vormgeving\\
\addlinespace
*Commercieel persoon* & Acquisitie\\
 & Transactierealisatie\\
 & Relatiemanagement\\
 & \\
 & \\
\addlinespace
 & \\
 & \\
 & \\
 & \\
 & \\
\bottomrule
\end{tabular}
\end{table}

\section{Onderwijsteam}\label{onderwijsteam}

\chapter{Inleiding}\label{inleiding}

In dit hoofdstuk vind je de doelen, de organisatie en het didactisch
opzet van de module.

\section{Doelen}\label{doelen}

Na voltooing van dit studietraject ben je in staat volwaardig mee te
werken binnen een ontwikkelteam voor \emph{Lead Generation}. Je kunt
verschillende rollen binnen het team vervullen en voegt waarde toe door
middel van je technische vaardigheden, je inzicht in sociale processen
en je professionele netwerk. Je bent bewust van je eigen passies en
krachten en hebt oog voor ethische en morele normen en waarden.

\section{Organisatie}\label{organisatie}

\subsection{Selectie}\label{selectie}

\subsection{Inschrijving}\label{inschrijving}

\subsection{Leertrajecten}\label{leertrajecten}

\subsubsection{Level 1 - Fundamentals}\label{level-1---fundamentals}

\subsubsection{Level 2 - Advanced}\label{level-2---advanced}

\subsubsection{Level 3 - Professional}\label{level-3---professional}

\subsubsection{Level 4 - Wizzard}\label{level-4---wizzard}

\section{Didactiek}\label{didactiek}

\chapter{Hello World}\label{hello-world}

In dit hoofdstuk vind je informatie over de voorbereiding op de module:

\section{Basiskennis}\label{basiskennis}

\section{Hardware}\label{hardware}

\section{Software}\label{software}

\section{Lesmaterialen}\label{lesmaterialen}

\section{Leeromgeving}\label{leeromgeving}

\section{Voortgang- en afronding}\label{voortgang--en-afronding}

\chapter{Competenties}\label{competenties}

In dit hoofdstuk vind je alles over de eindcompetenties van deze module.

Voor iedere teamrol geldt een aparte set van eindcompetenties. Voor
optimaal functioneren is het belangrijk dat je tenminste één rol op
level \emph{Professional} of \emph{Wizzard} ontwikkelt en voor alle
andere minimaal het level \emph{Fundamental} bereikt.

\section{Online Marketeer}\label{online-marketeer}

\begin{table}

\caption{\label{tab:unnamed-chunk-2}Overzicht van competenties en levels voor Online Marketeer}
\centering
\begin{tabular}[t]{lllll}
\toprule
Competentie & Fundamental & Advanced & Professional & Wizzard\\
\midrule
<i>Technisch</i> & a & c & a & a\\
 & b & d & b & c\\
<i>Team</i> & c & a & c & d\\
 & d & c & d & a\\
<i>Commercieel</i> & a & d & a & b\\
\addlinespace
 & b & c & b & c\\
<i>Maatschappelijk</i> & c & d & c & e\\
 &  &  &  & \\
 &  &  &  & \\
 &  &  &  & \\
\addlinespace
 &  &  &  & \\
 &  &  &  & \\
 &  &  &  & \\
 &  &  &  & \\
 &  &  &  & \\
\addlinespace
 &  &  &  & \\
 &  &  &  & \\
 &  &  &  & \\
 &  &  &  & \\
 &  &  &  & \\
 &  &  &  & \\
\bottomrule
\end{tabular}
\end{table}

\section{Content Marketeer}\label{content-marketeer}

\begin{table}

\caption{\label{tab:unnamed-chunk-3}Overzicht van competenties en levels voor Content Marketeer}
\centering
\begin{tabular}[t]{lllll}
\toprule
Competentie & Fundamental & Advanced & Professional & Wizzard\\
\midrule
<i>Technisch</i> & a & c & a & a\\
 & b & d & b & c\\
<i>Team</i> & c & a & c & d\\
 & d & c & d & a\\
<i>Commercieel</i> & a & d & a & b\\
\addlinespace
 & b & c & b & c\\
<i>Maatschappelijk</i> & c & d & c & e\\
 &  &  &  & \\
 &  &  &  & \\
 &  &  &  & \\
\addlinespace
 &  &  &  & \\
 &  &  &  & \\
 &  &  &  & \\
 &  &  &  & \\
 &  &  &  & \\
\addlinespace
 &  &  &  & \\
 &  &  &  & \\
 &  &  &  & \\
 &  &  &  & \\
 &  &  &  & \\
 &  &  &  & \\
\bottomrule
\end{tabular}
\end{table}

\section{Developer}\label{developer}

\begin{table}

\caption{\label{tab:unnamed-chunk-4}Overzicht van competenties en levels voor Developer}
\centering
\begin{tabular}[t]{lllll}
\toprule
Competentie & Fundamental & Advanced & Professional & Wizzard\\
\midrule
<i>Technisch</i> & a & c & a & a\\
 & b & d & b & c\\
<i>Team</i> & c & a & c & d\\
 & d & c & d & a\\
<i>Commercieel</i> & a & d & a & b\\
\addlinespace
 & b & c & b & c\\
<i>Maatschappelijk</i> & c & d & c & e\\
 &  &  &  & \\
 &  &  &  & \\
 &  &  &  & \\
\addlinespace
 &  &  &  & \\
 &  &  &  & \\
 &  &  &  & \\
 &  &  &  & \\
 &  &  &  & \\
\addlinespace
 &  &  &  & \\
 &  &  &  & \\
 &  &  &  & \\
 &  &  &  & \\
 &  &  &  & \\
 &  &  &  & \\
\bottomrule
\end{tabular}
\end{table}

\section{Designer}\label{designer}

\begin{table}

\caption{\label{tab:unnamed-chunk-5}Overzicht van competenties en levels voor Designer}
\centering
\begin{tabular}[t]{lllll}
\toprule
Competentie & Fundamental & Advanced & Professional & Wizzard\\
\midrule
<i>Technisch</i> & a & c & a & a\\
 & b & d & b & c\\
<i>Team</i> & c & a & c & d\\
 & d & c & d & a\\
<i>Commercieel</i> & a & d & a & b\\
\addlinespace
 & b & c & b & c\\
<i>Maatschappelijk</i> & c & d & c & e\\
 &  &  &  & \\
 &  &  &  & \\
 &  &  &  & \\
\addlinespace
 &  &  &  & \\
 &  &  &  & \\
 &  &  &  & \\
 &  &  &  & \\
 &  &  &  & \\
\addlinespace
 &  &  &  & \\
 &  &  &  & \\
 &  &  &  & \\
 &  &  &  & \\
 &  &  &  & \\
 &  &  &  & \\
\bottomrule
\end{tabular}
\end{table}

\section{Customer Relations}\label{customer-relations}

\begin{table}

\caption{\label{tab:unnamed-chunk-6}Overzicht van competenties en levels voor Customer Relations}
\centering
\begin{tabular}[t]{lllll}
\toprule
Competentie & Fundamental & Advanced & Professional & Wizzard\\
\midrule
<i>Technisch</i> & a & c & a & a\\
 & b & d & b & c\\
<i>Team</i> & c & a & c & d\\
 & d & c & d & a\\
<i>Commercieel</i> & a & d & a & b\\
\addlinespace
 & b & c & b & c\\
<i>Maatschappelijk</i> & c & d & c & e\\
 &  &  &  & \\
 &  &  &  & \\
 &  &  &  & \\
\addlinespace
 &  &  &  & \\
 &  &  &  & \\
 &  &  &  & \\
 &  &  &  & \\
 &  &  &  & \\
\addlinespace
 &  &  &  & \\
 &  &  &  & \\
 &  &  &  & \\
 &  &  &  & \\
 &  &  &  & \\
 &  &  &  & \\
\bottomrule
\end{tabular}
\end{table}

\chapter{Activiteiten}\label{activiteiten}

In dit hoofdstuk vind je een beschrijving van de verschillende
onderwijsactiviteiten.

\section{Weblectures}\label{weblectures}

\section{Workshops}\label{workshops}

\section{Individuele opdrachten}\label{individuele-opdrachten}

\section{Teamopdrachten}\label{teamopdrachten}

\section{Seminars en
bedrijfsbezoeken}\label{seminars-en-bedrijfsbezoeken}

\section{Maatschappelijke
participatie}\label{maatschappelijke-participatie}

\chapter{Coaching}\label{coaching}

Hier vind je informatie over begeleiding en samenwerking.

\bibliography{packages.bib,book.bib}


\end{document}
